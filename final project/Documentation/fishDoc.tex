\documentclass{scrartcl}
\usepackage[utf8]{inputenc}

\title{Principia Textilica:\\ Threads of Ancestry and Community}
\subtitle{Project Documentation}
\author{Felicitas Höbelt}
%\date{März 2015}

\usepackage{natbib}
\usepackage{graphicx}
\usepackage{caption}
\usepackage{subcaption}
%\usepackage{longtable}
%\usepackage{listings}
%here goes the settings for the formatting code-section, I just leave Matlab in as an example
%\lstset{language=Matlab, basicstyle=\footnotesize,
%keywords={break,case,catch,continue,else,elseif,end,for,function,
%global,if,otherwise,persistent,return,switch,try,while,ones,zeros}}
\graphicspath{ {./img/} }


\begin{document}

\maketitle

\section{Principia Textilica}

\subsection{Context}
In the course "Principia Textilica" we explored intersections of computational algorithms and textile craft, including pattern generation algorithms, weaving techniques, knot theory, hands-on experience with textile maschines, history lessons and more.
For the final project each student was encouraged to find a subject to implement some of the methods and techniques we got to see during the course.

\subsection{Motivation}

I study Computer Science and Media, which means my background is more algorithmic/technical than art-related.
Nevertheless I am familiar with a few textile techniques such as knotting (Macramé), weaving, sewing, felting and embroidery.\\
For inspiration I looked at two topics that have fascinated me for years: Modeling "Living" Things (plants/fungi/insects/animals) and complexity.\\
Complexity is often mistaken for complicatedness because the result looks complicated.
However the system that produces a complex result is itself very simple. It consists of a certain number of elements and a few elemantary rules which are applied to those elements.\\
My goal was to use both, complexity and modeling simple "life", to create a (textile) pattern.

\section{Implementation}

\subsection{Idea: The Pond/Fish tank}
I decided to go with a swarm model, which means I have a number of individuals which are controlled by a set of rules concerning themselves, their dynamic environment (e.g. the other members of the swarm) and their static environment (e.g. boundaries of the tank).
The metaphor I chose was a school of fish moving around in a pond or glass tank.

The desired pattern would be generated by the movements of the individuals. I wanted to start with the rules and continuously adapt them depending on the outcome.\\
I did not settle for a specific textile technique, at this point I only thought about assigning each of the individuals a thread. With this premise in mind, favourable textile techniques were embroidery and knotting.


\subsection{Rules}
My fish program is in 2D and has two main elements: the tank and the fish.

The behaviour of the swarm is mainly determined by four principles:\\
I Company - Each fish will follow other fish.\\
II Privacy - Each fish needs a minimum of private space.\\
III Security- Each (non-predatory) fish will avoid any predators.\\
IV Boundary- Each fish will stay inside the tank.\\

These swarm rules share some similarities with the "Boids" program by Craig Reynolds [ref], he developed an agent-based algorithm to simulate the flocking behaviour of birds.\\


The tank started out as rectangular, but I found a circle to be more aesthetically pleasing (as it is a softer, more natural shape) as well as more fitting for an embroidery hoop. The circle embodies the simple inside/outside-world that I need. The tank restricts the movement of the fish inside.
The fish swarm consists of individual fish that – exceptions are the color and the name – start out as equal clones. There are several more, but the most important fish parameters/properties are:
-	Size of field of view
-	Distance for avoiding other fish (Privacy)
-	Distance for following other fish (Company)
-	Speed for making turns
Each fish can be either harmless or predatory towards other fish.
There is no randomness in my program (again, except for the generation of colors).
-	If it is a predator or not.

\subsection{Code Samples}
Processing

\subsection{Parameter Variation + Observations + Images}

\subsection{Unused Parameters}


\section{Textile Implementation}

\subsection{Preparation?}
How I would translate that into something textile, I was not so sure at first, but I started with a thread-based idea, where each thread is seen as an individual. Macramé and Embroidery were favored candidates.

\subsection{}
\subsection{Finished Piece}

\section{}
\section{}
\subsection{}

%\begin{figure}[h]
%        \centering
%        \begin{subfigure}[b]{0.31\textwidth}
%                %\includegraphics[width=\textwidth]{dist1}
%        \end{subfigure}
%        \begin{subfigure}[b]{0.31\textwidth}
%                %\includegraphics[width=\textwidth]{dist2}
%        \end{subfigure}
%        %\caption{Visualization of the projective distorted points}
%\end{figure}





Trial/Error runs? Unused parameters?
References/inspiration/ideas:
-	Boids (Flocking simulation)
-	Pattern Generation
-	Changing Pattern in something fix
First (implemented) sketched out idea:
School of fish in a tank/pond, 2D to reduce scope of work:
-	Boids – idea for behaviour: seek companionship, avoid predators, walls and companions that get too close (company, safety, privacy)
-	(explain some parameters to finetune behaviour, also predators no real danger, like the problems we run from – they won’t kill us, but starve us emotionally)
-	(explain main mechanism: finding the direction vector in each step, no randomness involved)
-	Other alterations/ playing with ideas: start with one fish that can spawn children (keep on moving or die after spawning)

Intermediate result:
-	Tracking positions in pixels : image with traces on picture (pixels, not lines)
-	Image with connecting lines (data = lines)

Role of Textiles in this?
-	Threads/knots as small part with defined behaviour (1. idea)
-	Errors over time, two tanks with same starting population grow apart (2. Idea, did not quite work, funny because with handcraft that happens : no identical pieces)
-	Connections to ancestors (3. Idea, without predators)
-	Connections to ancestors should limit the "life" of the  fish somehow (3. idea) : threads also limited, no infinite resource in real life, limited freedom?
-	-> include death somehow? Maybe the thread becomes limited depending on how close you were when your ancestor died : shorter thread?
-	New connections to fish you are close to for a longer time : frienships/partnerships (different color/weight?)
Textile finished piece:
-	Embroidery: pixelated image to threaded image (boring)
-	Connections to threaded image(series) (better? Development of generations)
-	Tree (knots/perls? = fish, length of edge is also determined by fish-bowl-movement) / graph : flexible
TODO: startled spreads with a delay ("reaction time")


%----------------------------------------------------------------------------------------------------

%====================================================================================================
\newpage
\section*{Code}

%put the code here:
%\begin{lstlisting}
%\end{lstlisting}

%\bibliographystyle{plain}
%\bibliography{references}
\end{document}

